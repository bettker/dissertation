\newcommand{\mr}[2][noinline]{\todo[#1,fancyline,color=blue!20]{#2}}
\newcommand{\mri}[2][inline]{\todo[#1,fancyline,color=blue!20]{#2}}
\newcommand{\agp}[2][noinline]{\todo[color=orange!60,linecolor={orange!100},#1,fancyline,author=André]{#2}}
\newcommand{\agpi}[2][inline]{\todo[color=orange!60,linecolor={orange!100},#1,fancyline,author=André]{#2}}
\newcommand{\rv}[2][noinline]{\todo[color=red!50,linecolor={red!100},#1,fancyline,author=Rafael]{#2}}
\newcommand{\rvi}[2][inline]{\todo[color=red!50,linecolor={red!100},#1,fancyline,author=Rafael]{#2}}

\providecommand{\floor}[1]{\ensuremath{\left\lfloor #1\right\rfloor}}
\providecommand{\ceil}[1]{\ensuremath{\left\lceil #1\right\rceil}}

\providecommand{\multiset}[1]{\ensuremath{\{\!\!\{#1\}\!\!\}}}
\providecommand{\sas}{\ensuremath{\text{SAS}^{+}}\xspace}
\providecommand{\strips}{STRIPS\xspace}
\providecommand{\astar}{\ensuremath{\text{A}^{*}}\xspace}
\providecommand{\gbfs}{\ensuremath{\text{GBFS}}\xspace}
\providecommand{\wida}{\ensuremath{\text{W-IDA}^{*}}\xspace}
\providecommand{\ida}{\ensuremath{\text{IDA}^{*}}\xspace}
\providecommand{\h}{\ensuremath{h}\xspace}
\providecommand{\hvalue}[1]{\ensuremath{h^{#1}}\xspace}
\providecommand{\hff}{\hvalue{\text{FF}}}
\providecommand{\hgc}{\hvalue{\text{GC}}}
\providecommand{\hstar}{\hvalue{*}}
\providecommand{\hlmc}{\hvalue{\text{lm-c}}}
\providecommand{\hmax}{\hvalue{\text{max}}}
\providecommand{\hadd}{\hvalue{\text{add}}}
\providecommand{\hnn}{$\hat h$\xspace}
\providecommand{\hnrsl}{$\hat h^{\text{N-RSL}}$\xspace}
\providecommand{\hboot}{$\hat h^{\text{Boot}}$\xspace}
\providecommand{\hgc}{\hvalue{\text{gc}}}
\providecommand{\hhgn}{\hvalue{\text{HGN}}}
\providecommand{\unit}{/1\xspace}
\providecommand{\hvfc}{\text{SUI}\xspace}
\providecommand{\hmin}{\text{SAI}\xspace}
\providecommand{\rw}{{RW}\xspace}
\providecommand{\bfs}{{BFS}\xspace}
\providecommand{\dfs}{{DFS}\xspace}
\providecommand{\bfsrw}{\text{FSM}\xspace}
\providecommand{\nn}{{NN}\xspace}
\providecommand{\fssp}{{FSS}\xspace}
\providecommand{\bssp}{{BSS}\xspace}
\providecommand{\hnnrs}{$\hat h{^{20\%}_\text{\meanfx}}$\xspace}
\providecommand{\hnnrsfifty}{$\hat h{^{50\%}_\text{\meanfx}}$\xspace}
\providecommand{\hffexp}{$h^{FF}_{exp}$}
\providecommand{\hgcexp}{$h^{GC}_{exp}$}
\providecommand{\hnnbase}{$\hat h_{0}$\xspace}
\providecommand{\hnnbfs}{$\hat h_{\text{bfs}}$\xspace}
\providecommand{\hnndfs}{$\hat h_{\text{dfs}}$\xspace}
\providecommand{\hnnrw}{$\hat h_{\text{rw}}$\xspace}
\providecommand{\hnnbfsrw}{$\hat h_\text{fsm}$\xspace}
\providecommand{\hnnbfsrwl}[1]{\ensuremath{\hat h_{#1}}\xspace}
\providecommand{\hnnnomutex}{\ensuremath{\hat h^{'}}\xspace}
\providecommand{\hnnnomutexl}[1]{\ensuremath{\hat h^{'}_{#1}}\xspace}
\providecommand{\hnnrsp}[1]{\ensuremath{\hat h_\text{fsm}/^{#1\%}_{\text{RS}}}\xspace}
\providecommand{\hnnrslp}[2]{\ensuremath{\hat h_\text{fsm}^{#1}/^{#2\%}_{\text{RS}}}\xspace}
\providecommand{\define}[1]{#1}
\providecommand{\facts}{\ensuremath{L_F}\xspace}
\providecommand{\meanfx}{\ensuremath{L_{\overline{F}}}\xspace}
\providecommand{\default}{\ensuremath{L_{200}}\xspace}
\providecommand{\distfarthest}{\ensuremath{d^*}\xspace}

%% mathematical definitions
\ifcsname dom\endcsname\else\DeclareMathOperator{\dom}{dom}\fi
\DeclareMathOperator{\pre}{pre}
\DeclareMathOperator{\eff}{eff}
\DeclareMathOperator{\sucs}{succ}
\DeclareMathOperator{\pred}{pred}
\DeclareMathOperator{\functioninitial}{initial\_state}
\DeclareMathOperator{\functiongoal}{goal\_condition}
\DeclareMathOperator{\mutex}{mutex}
\DeclareMathOperator{\del}{del}
\DeclareMathOperator{\add}{add}
\ifcsname R\endcsname\else\newcommand{\R}{\ensuremath{\mathbb{R}}}\fi

\newtheorem{property}{Property}[section]

%% blocks world example
\providecommand{\facton}[2]{\ensuremath{\text{on}(#1,#2)}\xspace}
\providecommand{\factontable}[1]{\ensuremath{\text{on-table}(#1)}\xspace}
\providecommand{\factclear}[1]{\ensuremath{\text{clear}(#1)}\xspace}
\newcommand{\drawCube}[5]{
    \draw[#4!90, fill=#4!50] (#1,#2,#3+#5) -- ++(0,#5,0) -- ++(#5,0,0) -- ++(0,-#5,0) -- cycle; % front
    \draw[#4!90, fill=#4!50] (#1,#2+#5,#3) -- ++(0,0,#5) -- ++(#5,0,0) -- ++(0,0,-#5) -- cycle; % top
    \draw[#4!90, fill=#4!50] (#1+#5,#2,#3) -- ++(0,#5,0) -- ++(0,0,#5) -- ++(0,-#5,0) -- cycle; % right
}
