\chapter{Preliminaries}\label{sec:preliminaries}

\rvi{WIP from here until the end of the chapter}

\section{Background and Notation}

A~\sas planning task~\cite{Backstrom.Nebel/1995} is a tuple $\Pi=\langle\mathcal{V},\mathcal{O},s_0,s^*, \text{cost}\rangle$, where $\mathcal{V}$~is a set of variables, $\mathcal{O}$~is a set of operators, $s_0$~an initial state, $s^*$ a goal condition, and $\text{cost}:\mathcal{O}\rightarrow\R_{+}$ a function mapping operators to costs. A variable $v\in \mathcal{V}$ has a finite domain~$D(v)$, and a partial state~$s$ is a partial function $s:\mathcal{V}\rightarrow \mathcal{D}$, with $\mathcal{D}=\cup_{v\in \mathcal{V}}D(v)$ and such that $s(v)\in D(v)$, whenever $s(v)$ is defined. Otherwise, $s(v)$ is undefined, which we also write as $s(v)=\bot$. A (complete) state $s$ is a partial state defined on all variables in~$\mathcal{V}$. The initial state~$s_0$ is a state, and the goal condition~$s^*$ is a partial state. A partial state can be interpreted as a set of facts where facts are pairs $(v,s(v))$ with $s(v)\in D(v) \cup \{\bot\}$, where the undefined variables assume all possible values in their domain. An operator~$o\in \mathcal{O}$ is a pair of preconditions and effects $(\pre(o),\eff(o))$, both partial states. Operator~$o$ can be applied to state~$s$ if $\pre(o)\subseteq s$ (if $s$ is a partial state, when $\pre(o)$ mentions a variable $v$ and $s(v) = \bot$ then it is applicable over $v$), and its application produces a successor state $s'=\sucs(s,o):=\eff(o)\circ s$, where $s'=t\circ s$ is defined by $s'(v)=t(v)$ for all $v$ such that $t(v)$ is defined, and $s'(v)=s(v)$ otherwise. The set of all successor states of state $s$ is~$\sucs(s)=\{\sucs(s,o)\mid o\in \mathcal{O}, \text{o applicable to }s\}$.  A sequence of operators $\pi=(o_1,\ldots,o_k)$, $o_i\in \mathcal{O}$ is valid for initial state~$s_0$ if for $i\in[k]$ operator~$o_i$ can be applied to $s_{i-1}$ and produces $s_i=\sucs(s_{i-1},o_i)$. A \define{plan} is a valid sequence~$\pi$ for~$s_0$ such that~$s_k\in s^*$ and the cost of plan~$\pi$ is $\sum_{i\in[k]} \text{cost}(o_i)$.

The predecessor $\pred(s,o)$ of a partial state $s$ under operator $o$ is obtained by \define{regression}. For regression, we consider an operator $o\in \mathcal{O}$ to be relevant for partial state~$s$ if $\eff_r=\dom(\eff(o))\cap\dom(s)\neq\emptyset$; the operator is consistent if $\eff(o)|_{\eff_r} \subseteq s$. Relevance requires at least one defined effect in the partial state to be regressed, consistency and an agreement on defined effects. An operator~$o$ then is \define{backward} applicable in partial state~$s$ if it is relevant and consistent with~$s$ and leads to predecessor $r=\pre(o)\circ (s|_{\dom(s)\setminus\eff_r})$. Note that $\sucs(r,o)\subseteq s$, but may differ from $s$. Similar to progression, a partial state~$s$ has predecessors $\pred(s)=\{\pred(s,o)\mid o\in \mathcal{O}, \text{o backward applicable to }s\}$. A regression sequence from state $s_0$
then is valid if $o_i$ can be applied to $s_{i-1}$ and produces $s_i=\pred(s_{i-1},o_i)$. All partial states~$s_k$ can reach a partial state $s\subseteq s_0$ in at most~$k$ forward applications of the reversed operator sequence. Since the goal is a partial state, a valid regression sequence $\rho=(o_1,\ldots,o_k)$ will generate a sequence of partial states that can reach the goal in at most $k$ steps and with a cost at most $\sum_{i\in[k]}\text{cost}(o_i)$.

Given a planning task, the set of all states over $\mathcal{V}$ is the state space. The \define{forward state space}~(\fssp) is the set of all states reachable from the initial state~$s_0$ by applying a sequence of operators. Similarly, the \define{backward state space}~(\bssp) is the set of all partial states reachable from the goal~$s^*$ by applying a sequence of backward applicable operators. 

\section{Learning Heuristic Functions with Neural Networks}
\label{sec:learning}

Many heuristics for classical planning are derived from a model of the task, such as the \sas model introduced in the previous section. An obvious alternative is to learn to map a state $s$ to its heuristic value $h(s)$. We focus here on learning with NNs, although other supervised learning methods could be used. To learn a heuristic function, an NN is trained on pairs of states~$s$ and cost-to-goal estimates~$c$. The learned heuristic functions are usually not admissible, so traditional optimality guarantees are lost.

A propositional representation of a state is more suitable for learning functions over states. To this end, consider a planning task $\Pi=\langle\mathcal{V},\mathcal{O},s_0,s^*, \text{cost}\rangle$, and let $\mathcal{V}=\{v_1,\ldots,v_n\}$ and $D(v_i)=\{d_{i1},\ldots,d_{i,s_i}\}$, $i\in[n]$ be some order of the variables and their domains. We represent any state $s$ by a sequence of facts $$\mathcal{F}(s)=(f_{11},f_{12},\ldots,f_{1,s_1},\ldots,f_{n1},f_{n2},\ldots,f_{n,s_n}),$$ where each fact $f_{ij}=[s(v_i)=d_{ij}]$ indicates if variable $v_i$ assumes value $d_{ij}$ in state $s$. Note that facts $\mathcal{F}_i=\{f_{i1},\ldots,f_{i,s_i}\}$ corresponding to variable $v_i$ satisfy the consistency condition $\sum_{f\in \mathcal{F}_i} f\leq 1$ since each variable assumes at most one value, and $\sum_{f\in \mathcal{F}_i} f=0$ only if $v_i$ is undefined. More generally, for any set of propositions $\mathcal{P}$ we write $\mutex(\mathcal{P})$ if $\sum_{p\in \mathcal{P}} [p]\leq 1$ must be satisfied in states of $\Pi$. Many planning systems can deduce mutexes from the description of the planning task $\Pi$~\cite{Helmert/2009}; we will discuss and analyze their utility for sampling states later. Some architectures provide additional input to the NN, e.g.,~the propositional representation of the goal condition. The target output for training may be the cost-to-goal estimates directly or some encoding of them.

An important aspect of sample generation related to challenges C1 and C2 (Section \ref{sec:introduction}) is the degree of dependency on the domain model or the planning task and the cost of generating the samples. Ideally, we would like to learn in a \define{model-free} setting where we interact with the planning task only by functions that allow accessing the initial state~$s_0$, the goal condition~$s^*$, and the applicable operators in a partial state. In this setting, we do not have access to the logical description of operators, we only have access to \define{black-box functions}~\cite{Sturtevant2019} -- which could also be learned -- that, given a partial state, returns its successors and predecessors. 
Note that this setting is also used in reinforcement learning.
Some approaches also provide access to each variable's domain. In contrast, a \define{model-based} heuristic uses the complete description of the model, which permits, for example, reasoning about operators and the computation of mutexes. Finally, we consider a heuristic \define{strongly model-based} if it uses complete planners to generate the samples, including search algorithms and logic-based heuristic functions.

The cost of sample generation depends on the number of samples and the cost to generate each. This generates the problem of deciding how many samples are required, although in general, only a very small part of the state space can be sampled. More importantly, ideal samples would be labeled with the perfect heuristic $h^*$, which maps each state~$s$ to the cost of an optimal $s$-plan or $\infty$ if no such plan exists. 
In general, ideal labeling is impractical since it requires solving the planning task on a large number of initial states. Therefore, we are mainly interested in good heuristic estimates that can be generated fast. We analyze the influence of sample size and quality experimentally later.

Additionally, and related to challenges C3 and C4, network architecture and sample generation depends on the range of tasks the learner intends to generalize. This may be the state space of a planning task, a planning domain, or an entire planning formalism. In the first case, the set of planning tasks is defined over any pair of initial state $s_0$ and goal $s^*$. Often the set of planning tasks is restricted to select the initial state from the \fssp of some given initial state and to a fixed goal. In the second case, the learned function has to generalize over all domain tasks. Finally, a learning-based heuristic that generalizes over a planning formalism is domain-independent. An important aspect of sample generation is the distribution of states part of the sample set. For example, the sample set can contain only states with a short distance to the goal or only states with a short distance to the initial states. In our study, the distribution of states describes how often each \hstar-value occurs in the sample set.
 
\section{Related Work}
\label{sec:related}

There have been two main research foci on learning heuristic functions. The first uses strongly model-based approaches to generate samples and aims to generalize over domains or a planning formalism. The second uses partially model-based approaches to generate samples and aims to generalize only over a state space. We describe these approaches as partially model-based because they only use the model to identify mutexes.

The usual setting for the first set of approaches~\cite{Toyer.etal/2018,Toyer.etal/2020,Shen.etal/2020,Gehring.etal/2022,Stahlberg.etal/2022} is to train a structured NN with samples of small tasks of a domain generated with a strongly model-based method and evaluated on larger tasks of the same domain.  
The structured networks trained can be general networks such as neural logic machines~\cite{Dong.etal/2018} and graph neural networks~\cite{Gori.etal/2005,Scarselli.etal/2008}, or networks proposed explicitly in the context of planning such as hypergraph neural network~\cite{Shen.etal/2020} and Action Schema Networks~\cite{Toyer.etal/2018}.
These networks require the logical description of the domain and the task to be instantiated and can generalize well, with competitive results compared to logic-based heuristics.
These approaches also help in understanding learning heuristics. For example, the main goal of \citeyear{Stahlberg.etal/2022} is to understand the expressive power and limitations of learning heuristics.
The main limitation of these approaches is the strong dependence on the domain model and task description.
  
The second set of approaches~\cite{Ferber.etal/2020a, Yu.etal/2020, Ferber.etal/2022, OToole/2022} typically trains an FNN and evaluates the learned heuristic on a state space using tasks with the same goal and different initial states. 
These networks are trained with pairs of states and cost-to-goal estimates. 
In this setting, \citeyear{Ferber.etal/2020a} systematically study hyperparameters on the architecture of the FNN and found that their influence is secondary. They found that for a fixed architecture, two aspects significantly influence how informed the heuristic is: the subset of selected samples and the size of the sample set. 

Furthermore, \citeyear{Yu.etal/2020} and \citeyear{OToole/2022} perform backward searches from the goal, the former with depth-first search and the latter with random walks. Both use the lowest depth in which the state was generated as cost-to-goal estimates. \citeyear{Ferber.etal/2022} uses a combination of backward and forward searches~\cite{Arfaee.etal/2011}. First, it generates new initial states with backward random walks and then solves them with a GBFS guided by a learned heuristic. The plans found provide the samples, and each sample is a state in the plan with the cost-to-goal estimate as its distance to the goal.
\citeyear{OToole/2022} also presented a method to generate samples that do not use expansions. This method includes randomly generated states in the sample set with high cost-to-goal estimates.
\citeyear{OToole/2022} showed that this method substantially increases coverage. 
The methods from the second set of approaches are highly independent of the domain model and planning task description and require low computational resources to generate samples and train the FNN. However, they suffer from high performance variability, given the FNN initialization and the sample set used. Also, despite having competitive results compared to logic-based heuristics, they are still unable to surpass the goal-count heuristic.
